\documentclass{article}
\usepackage{graphicx} % Required for inserting images

\title{Relatório}
\author{Júlio César Gonzaga Ferreira Silva}
\date{August 2024}

\begin{document}

\maketitle

\section{Descrição do que foi feito:}
 O objetivo deste programa é gerar e imprimir todos os subgrafos possíveis de um grafo completo com n vértices, onde n é informado pelo usuário. Além disso, o programa informa o número total de subgrafos gerados. A priori pedi para o número de vértices para descobrir o número de arestas, que é n(n-1) / 2, onde n é o numero de vértices.
 
 Metodologia Utilizada: O programa gera todas as possíveis arestas do grafo completo utilizando dois loops alinhados. Cada aresta é representada como um par de inteiros, onde o primeiro inteiro é o vértice de inicial e o segundo é o vértice de destino. Essas arestas são armazenadas em um vetor de pares (vector<pair<int, int>>).

Geração de Subgrafos:
Para gerar todos os subgrafos, utilizamos uma abordagem baseada em máscaras de bits. Como o número total de arestas é numArestas, cada subgrafo pode ser representado por uma combinação das arestas presentes.

Cada subgrafo é gerado iterando sobre todas as possíveis máscaras de bits de comprimento numArestas.
Se o k-ésimo bit da máscara estiver ativado, a aresta correspondente é incluída no subgrafo.
Impressão dos Subgrafos:
Para cada máscara de bits, o programa imprime o subgrafo correspondente, exibindo todas as arestas que estão presentes nele.

\end{document}
